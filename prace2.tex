\documentclass{article}

\usepackage[czech]{babel}

% pro specifikaci formátu datumu
\usepackage{datetime}
\newdateformat{czdate}{\THEDAY.\THEMONTH.\:\THEYEAR}
% použití: \czdate\today

% definice zadani
\newenvironment{zadani}
{\begin{center}\textbf{Zadání:}\itshape}{\end{center}}


\begin{document}

\title{Samostatná práce II – BPC-EMV2}
\author{Jakub Charvot}
\date{\czdate\today}
\maketitle

\section{Úvod}

V této práci jsou prezentovány výpočty z oblasti elektronových procesů, iontových procesů, rentgenových procesů, jaderných procesů, laserových procesů a ultraakustických procesů.

\section{Elektronové procesy}

\subsection{Přehled elektronových procesů}

Elektronové procesy jsou procesy, při kterých jsou zapojeny elektrony. Tyto procesy mohou být popsány pomocí různých metod, např. Bornovy aproximace nebo Hartreeho-Fockovy metody.

\subsection{Příklad 1.1.2 - výpočet}
\begin{zadani}
    Určete hustotu emisního proudu rozžhaveného wolframového vlákna 
    při teplotě 2 227 °C , působí-li současně elektrické pole o intenzitě 
    E = 2.105 V.m-1
\end{zadani}

V této sekci je uveden výpočet pro příklad 1.1.2. Jsou uvedeny použité metody a vypočtené výsledky.

\subsection{Příklad 1.1.3 - výpočet}
\begin{zadani}
    Určete emisní proud katody vyrobené ve tvaru disku o průměru 10 mm
    z wolframu, vyhřáté na teplotu 2 227 oC. Urychlovací anoda je umístěna 
    ve vzdálenosti  50 mm  od katody a je na ni přivedeno napětí  10 kV
\end{zadani}

V této sekci je uveden výpočet pro příklad 1.1.3. Jsou uvedeny použité metody a vypočtené výsledky.

\section{Iontové procesy}

\subsection{Přehled iontových procesů}

Iontové procesy jsou procesy, při kterých jsou zapojeny ionty. Tyto procesy mohou být popsány pomocí různých metod, např. Bornovy aproximace nebo Hartreeho-Fockovy metody.

\subsection{Příklad 1.2.1 - výpočet}
\begin{zadani}
    Srovnejte základní vlastnosti elektronů a iontů.
\end{zadani}


V této sekci je uveden výpočet pro příklad 1.2.1. Jsou uvedeny použité metody a vypočtené výsledky.

\subsection{Příklad 1.2.2 - výpočet}
\begin{zadani}
    Stanovte maximální koncentraci implantovaných iontů bóru do 
    monokrystalu křemíku ve vzdálenosti středního doletu iontů  Rp od 
    povrchu destičky a koncentraci iontů ve vzdálenosti  Rp \(\pm \Delta\)Rp a na 
    povrchu monokrystalu.  \(\Delta\)Rp  je střední kvadratická odchylka. Zjištěný  
    koncentrační profil naznačte graficky. Celková dávka  Q = 1018 m-2 při  
    energie iontů  100 keV
\end{zadani}


V této sekci je uveden výpočet pro příklad 1.2.2. Jsou uvedeny použité metody a vypočtené výsledky.

\section{Rentgenové procesy}

\subsection{Přehled rentgenových procesů}

Rentgenové procesy jsou procesy, při kterých jsou zapojeny rentgenové záření. Tyto procesy mohou být popsány pomocí různých metod, např. Bornovy aproximace nebo Hartreeho-Fockovy metody.

\subsection{Příklad 2.1.1 - výpočet}
\begin{zadani}

\end{zadani}


V této sekci je uveden výpočet pro příklad 2.1.1. Jsou uvedeny použité metody a vypočtené výsledky.

\subsection{Příklad 2.1.2 - výpočet}
\begin{zadani}

\end{zadani}


V této sekci je veden výpočet pro příklad 2.1.2. Jsou uvedeny použité metody a vypočtené výsledky.

\section{Jaderné procesy}

\subsection{Přehled jaderných procesů}

Jaderné procesy jsou procesy, při kterých jsou zapojena jádra atomů. Tyto procesy mohou být popsány pomocí různých metod, např. Bornovy aproximace nebo Hartreeho-Fockovy metody.

\subsection{Příklad 2.2.1 - výpočet}
\begin{zadani}

\end{zadani}


V této sekci je uveden výpočet pro příklad 2.2.1. Jsou uvedeny použité metody a vypočtené výsledky.

\subsection{Příklad 2.2.7 - výpočet}
\begin{zadani}

\end{zadani}


V této sekci je uveden výpočet pro příklad 2.2.7. Jsou uvedeny použité metody a vypočtené výsledky.

\section{Laserové procesy}

\subsection{Přehled laserových procesů}

Laserové procesy jsou procesy, při kterých jsou zapojeny lasery. Tyto procesy mohou být popsány pomocí různých metod, např. Bornovy aproximace nebo Hartreeho-Fockovy metody.

\subsection{Příklad 3.1.1 - výpočet}
\begin{zadani}

\end{zadani}


V této sekci je uveden výpočet pro příklad 3.1.1. Jsou uvedeny použité metody a vypočtené výsledky.

\subsection{Příklad 3.1.4 - výpočet}
\begin{zadani}

\end{zadani}


V této sekci je uveden výpočet pro příklad 3.1.4. Jsou uvedeny použité metody a vypočtené výsledky.

\section{Ultraakustické procesy}

\subsection{Přehled ultraakustických procesů}

Ultraakustické procesy jsou procesy, při kterých jsou zapojeny zvukové vlny. Tyto procesy mohou být popsány pomocí různých metod, např. Bornovy aproximace nebo Hartreeho-Fockovy metody.

\subsection{Příklad 3.2.4 - výpočet}
\begin{zadani}

\end{zadani}


V této sekci je uveden výpočet pro příklad 3.2.4. Jsou uvedeny použité metody a vypočtené výsledky.

\subsection{Příklad 3.2.7 - výpočet}
\begin{zadani}

\end{zadani}


V této sekci je uveden výpočet pro příklad 3.2.7. Jsou uvedeny použité metody a vypočtené výsledky.

\section{Závěr}

V této práci byly prezentovány výpočty z oblasti elektronových procesů, iontových procesů, rentgenových procesů, 
jaderných procesů, laserových procesů a ultraakustických procesů. Pro každou oblast byly uvedeny příklady a provedeny výpočty s použitím různých metod. Výsledky těchto výpočtů byly uvedeny v příslušných sekcích.

Výpočty byly provedeny s ohledem na principy dané oblasti a s použitím vhodných metod. Výsledky ukázaly, že metoda použitá k výpočtu může mít významný vliv na výsledky a jejich správnost.

Tato práce by mohla být rozšířena o další příklady a oblasti, a tím poskytnout více informací o různých procesech. Zároveň by bylo možné věnovat se více srovnání různých metod a zhodnocení, která metoda je pro danou oblast nejvhodnější.



\end{document}

