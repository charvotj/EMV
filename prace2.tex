\documentclass[a4paper,onecolumn,12pt]{article}

\usepackage[czech]{babel}
\usepackage[utf8]{inputenc}
\usepackage[left=2.5cm,right=2.5cm,top=2.5cm,bottom=2.5cm]{geometry}

% pro specifikaci formátu datumu
\usepackage{datetime}
\newdateformat{czdate}{\THEDAY.\THEMONTH.\:\THEYEAR}
% použití: \czdate\today

% definice zadani
\newenvironment{zadani}
{\begin{center}\textbf{Zadání:}\itshape}{\end{center}}

% odstavce
\usepackage{parskip}

%====== Units =====
\usepackage{siunitx}
\sisetup{inter-unit-product =\ensuremath{\cdot}}
\sisetup{group-digits = integer}
\sisetup{group-minimum-digits = 4}
\sisetup{group-separator=\,}
\sisetup{output-decimal-marker = {,}}
\sisetup{exponent-product = \ensuremath{\cdot}}
\sisetup{separate-uncertainty}
\sisetup{tight-spacing = false}
%\sisetup{scientific-notation = true}
%\sisetup{round-mode=places,round-precision=4}
%\sisetup{evaluate-expression}


% Boltzmannova konstanta
\newcommand{\constkval}{\num{1.38e-23}}
\newcommand{\constkunit}{\unit{\joule\per\kelvin}}

% Planckova konstanta
\newcommand{\consthval}{\num{6.626e-34}}
\newcommand{\consthunit}{\unit{\joule\second}}

% Rychlost světla
\newcommand{\constcval}{\num{2.998e8}}
\newcommand{\constcunit}{\unit{\meter\per\second}}

% Elementární náboj
\newcommand{\consteval}{\num{1.602e-19}}
\newcommand{\consteunit}{\unit{\coulomb}}

% Hmotnost elektronu
\newcommand{\constmeval}{\num{9.109e-31}}
\newcommand{\constmeunit}{\unit{\kilogram}}

% align bloky rovnic
\usepackage{amsmath}

\begin{document}

\title{Samostatná práce II – BPC-EMV2}
\author{Jakub Charvot}
\date{\czdate\today}
\maketitle

\section*{Úvod}

V této práci jsou prezentovány výpočty z oblasti elektronových procesů, iontových procesů, rentgenových procesů, jaderných procesů, laserových procesů a ultraakustických procesů.


% Elektronove procesy
\section{Elektronové procesy}

\subsection{Přehled elektronových procesů}

Elektronové procesy jsou procesy, při kterých jsou zapojeny elektrony. Tyto procesy mohou být popsány pomocí různých metod, např. Bornovy aproximace nebo Hartreeho-Fockovy metody.

\subsection{Příklad 1.1.2 - výpočet}
\begin{zadani}
    Určete hustotu emisního proudu rozžhaveného wolframového vlákna 
    při teplotě 2 227 °C , působí-li současně elektrické pole o intenzitě 
    E = 2.105 V.m-1
\end{zadani}

V této sekci je uveden výpočet pro příklad 1.1.2. Jsou uvedeny použité metody a vypočtené výsledky.

\subsection{Příklad 1.1.3 - výpočet}
\begin{zadani}
    Určete emisní proud katody vyrobené ve tvaru disku o průměru 10 mm
    z wolframu, vyhřáté na teplotu 2 227 oC. Urychlovací anoda je umístěna 
    ve vzdálenosti  50 mm  od katody a je na ni přivedeno napětí  10 kV
\end{zadani}

V této sekci je uveden výpočet pro příklad 1.1.3. Jsou uvedeny použité metody a vypočtené výsledky.

% Iontove procesy
\section{Iontové procesy}

\subsection{Přehled iontových procesů}

Iontové procesy jsou procesy, při kterých jsou zapojeny ionty. Tyto procesy mohou být popsány pomocí různých metod, např. Bornovy aproximace nebo Hartreeho-Fockovy metody.

\subsection{Příklad 1.2.1 - výpočet}
\begin{zadani}
    Srovnejte základní vlastnosti elektronů a iontů.
\end{zadani}


V této sekci je uveden výpočet pro příklad 1.2.1. Jsou uvedeny použité metody a vypočtené výsledky.

\subsection{Příklad 1.2.2 - výpočet}
\begin{zadani}
    Stanovte maximální koncentraci implantovaných iontů bóru do 
    monokrystalu křemíku ve vzdálenosti středního doletu iontů  Rp od 
    povrchu destičky a koncentraci iontů ve vzdálenosti  Rp \(\pm \Delta\)Rp a na 
    povrchu monokrystalu.  \(\Delta\)Rp  je střední kvadratická odchylka. Zjištěný  
    koncentrační profil naznačte graficky. Celková dávka  Q = 1018 m-2 při  
    energie iontů  100 keV
\end{zadani}
t

V této sekci je uveden výpočet pro příklad 1.2.2. Jsou uvedeny použité metody a vypočtené výsledky.

% Rentgenove procesy
\section{Rentgenové procesy}

\subsection{Přehled rentgenových procesů}

Rentgenové procesy jsou procesy, při kterých jsou zapojeny rentgenové záření. Tyto procesy mohou být popsány pomocí různých metod, např. Bornovy aproximace nebo Hartreeho-Fockovy metody.

\subsection{Příklad 2.1.1 - výpočet}
\begin{zadani}

\end{zadani}


V této sekci je uveden výpočet pro příklad 2.1.1. Jsou uvedeny použité metody a vypočtené výsledky.

\subsection{Příklad 2.1.2 - výpočet}
\begin{zadani}
    Stanovte nejkratší vlnovou délku rentgenového záření vzniklého 
    po dopadu svazku elektronů na kov. Urychlovací napětí  U = 2 kV
\end{zadani}


V této sekci je veden výpočet pro příklad 2.1.2. Jsou uvedeny použité metody a vypočtené výsledky.


% Jaderné procesy
\section{Jaderné procesy}

\subsection{Přehled jaderných procesů}

Jaderné procesy jsou procesy, při kterých jsou zapojena jádra atomů. Tyto procesy mohou být popsány pomocí různých metod, např. Bornovy aproximace nebo Hartreeho-Fockovy metody.

\subsection{Příklad 2.2.1 - výpočet}
\begin{zadani}

\end{zadani}


V této sekci je uveden výpočet pro příklad 2.2.1. Jsou uvedeny použité metody a vypočtené výsledky.

\subsection{Příklad 2.2.7 - výpočet}
\begin{zadani}

\end{zadani}


V této sekci je uveden výpočet pro příklad 2.2.7. Jsou uvedeny použité metody a vypočtené výsledky.



% Laserove procesy
\section{Laserové procesy}

\subsection{Přehled laserových procesů}

Laserové procesy jsou procesy, při kterých jsou zapojeny lasery. Tyto procesy mohou být popsány pomocí různých metod, např. Bornovy aproximace nebo Hartreeho-Fockovy metody.

\subsection{Příklad 3.1.1 - výpočet}
\begin{zadani}

\end{zadani}


V této sekci je uveden výpočet pro příklad 3.1.1. Jsou uvedeny použité metody a vypočtené výsledky.

\subsection{Příklad 3.1.4 - výpočet}
\begin{zadani}

\end{zadani}


V této sekci je uveden výpočet pro příklad 3.1.4. Jsou uvedeny použité metody a vypočtené výsledky.


% Ultraakustické procesy
\section{Ultraakustické procesy}

\subsection{Přehled ultraakustických procesů}

Ultraakustické procesy jsou procesy, při kterých jsou zapojeny zvukové vlny. Tyto procesy mohou být popsány pomocí různých metod, např. Bornovy aproximace nebo Hartreeho-Fockovy metody.

\subsection{Příklad 3.2.4 - výpočet}
\begin{zadani}
    Jak velký výkon bude předán destičce o ploše 1 cm2 ponořené ve 
    vodě, bude-li amplituda kmitů Y = 10-6 m  při kmitočtech 20 kHz a 
    1 MHz, při kolmém dopadu a úplné absorpci vlnění?
\end{zadani}


Výkon akustických vln závisí na prostředí, kde vlna působí (to určuje rychlost šíření vlnění) a také jeho amplituda. Výkon označíme \(P\) a vyjdeme z následujícího vztahu:
\begin{align*}
    P &= \frac{1}{2}\omega^2 Y^2 \rho vS
\end{align*}
Dosazení pro \qty{20}{kHz}:
\begin{align*}
    P &= \frac{1}{2}(2\pi\cdot \num{20e3})^2 (\num{e-6})^2 \num{1000}\cdot \num{1483}\cdot \num{1e-4} \\
    P &= \qty{1,17}{\watt} \\
\end{align*}
Dosazení pro \qty{1}{MHz}:
\begin{align*}
    P &= \frac{1}{2}(2\pi\cdot \num{1e6})^2 (\num{e-6})^2 \num{1000}\cdot \num{1483}\cdot \num{1e-4} \\
    P &= \qty{2,926}{\kilo\watt} \\
\end{align*}

\subsection{Příklad 3.2.7 - výpočet}
\begin{zadani}
    Vypočtěte jak velká část intenzity dopadajícího ultrazvukového vlnění 
    projde do oceli a jak velká část se odrazí při kolmém dopadu 
    akustické vlny na rozhraní  voda – ocel,  resp.  vzduch – ocel
\end{zadani}


Pro rozhraní libovolných dvou prostředí lze stanovit součinitel odrazu, který udává, jak velká část intenzity vlnění se odrazí zpět, zbytek naopak projde. 
Tento součinitel závisí na akustickém vlnovém odporu obou prostředí. 

Výpočet akustického vlnového odporu:
\[
    Z_{a} = \rho v
\]
\begin{align*}
  \text{Vz}&\text{duch:} & \text{Vo}&\text{da:} & \text{Oc}&\text{el:} \\
  Z_{vz} &= \rho_{vz} v_{vz} & Z_{vo} &= \rho_{vo} v_{vo} & Z_{oc} &= \rho_{oc} v_{oc} \\
  Z_{vz} &= \num{1,276}\cdot\num{332} & Z_{vo} &= \num{1000}\cdot\num{1483} & Z_{oc} &= \num{7800}\cdot\num{6000} \\
  Z_{vz} &= \qty{423.6}{\kilo\gram\per\square\meter\per\second} & Z_{vo} &= \qty{1.483e6}{\kilo\gram\per\square\meter\per\second} & Z_{oc} &= \qty{4.68e7}{\kilo\gram\per\square\meter\per\second} \\
\end{align*}

Koeficienty odrazu:
\begin{align*}
  R_{vo-oc} &= \left( \frac{Z_{vo}-Z_{oc}}{Z_{vo}+Z_{oc}} \right)^2 \\
  R_{vo-oc} &= \left( \frac{\num{1,483e6}-\num{4,68e7}}{\num{1,483e6}+\num{4,68e7}} \right)^2 \\
  R_{vo-oc} &= \num{0,881} = \qty{88,1}{\percent}  
\end{align*}
Na rozhraní voda--ocel se odrazí \qty{88,1}{\percent} intenzity vlnění, zbylých \qty{11,9}{\percent} projde dále.

\begin{align*}
    R_{vo-oc} &= \left( \frac{Z_{vz}-Z_{oc}}{Z_{vz}+Z_{oc}} \right)^2 \\
    R_{vo-oc} &= \left( \frac{\num{423,6}-\num{4,68e7}}{\num{423,6}+\num{4,68e7}} \right)^2 \\
    R_{vo-oc} &= \num{1} = \qty{100}{\percent}  
\end{align*}
Na rozhraní vzduch--ocel se odrazí téměř \qty{100}{\percent} intenzity vlnění.



\section*{Závěr}

V této práci byly prezentovány výpočty z oblasti elektronových procesů, iontových procesů, rentgenových procesů, 
jaderných procesů, laserových procesů a ultraakustických procesů. Pro každou oblast byly uvedeny příklady a provedeny výpočty s použitím různých metod. Výsledky těchto výpočtů byly uvedeny v příslušných sekcích.

Výpočty byly provedeny s ohledem na principy dané oblasti a s použitím vhodných metod. Výsledky ukázaly, že metoda použitá k výpočtu může mít významný vliv na výsledky a jejich správnost.

Tato práce by mohla být rozšířena o další příklady a oblasti, a tím poskytnout více informací o různých procesech. Zároveň by bylo možné věnovat se více srovnání různých metod a zhodnocení, která metoda je pro danou oblast nejvhodnější.



\end{document}


