\section{Rentgenové procesy}

\subsection{Přehled rentgenových procesů}

Rentgenové procesy jsou procesy, při kterých jsou zapojeny rentgenové záření. Tyto procesy mohou být popsány pomocí různých metod, např. Bornovy aproximace nebo Hartreeho-Fockovy metody.

\subsection{Příklad 2.1.1 - výpočet}
\begin{zadani}
    Stanovte nejkratší vlnovou délku rentgenového záření vzniklého 
    po dopadu svazku elektronů na kov. Urychlovací napětí  U = 2 kV
\end{zadani}


Nejkratší vlnovou délku bude mít záření s nejvyšší energií, musíme tedy stanovit maximální energii dopadajících elektronů. Ty jsou urychlovány napětím \(U\), bude to tedy vypadat takto:
\begin{align*}
  \lambda_{min} &=\frac{hc}{E}=\frac{hc}{qU} \\
  \lambda_{min} &=\frac{\consthval\cdot\constcval}{\consteval\cdot \num{2e3}}\\
  \lambda_{min} &= \qty{0,619}{nm}
\end{align*}


\subsection{Příklad 2.1.2 - výpočet}
\begin{zadani}
    Stanovte vlnovou délku rentgenového záření pro kterou dosáhne 
    intenzita záření maxima. Urychlovací napětí  U = 2 kV
\end{zadani}


Ze spektrální charakteristiky vyzářeného rtg. záření bylo empiricky zjištěno, že maximální intenzita záření odpovídá frekvenci, při které má záření energii připbližně 0,6 maximální hodnoty. Upravíme tedy výpočet:

\begin{align*}
    \lambda_{I_{max}} &=\frac{hc}{0,6E}=\frac{hc}{0,6qU} \\
    \lambda_{I_{max}} &=\frac{\consthval\cdot\constcval}{0,6\consteval\cdot \num{2e3}}\\
    \lambda_{I_{max}} &= \qty{1,032}{nm}
  \end{align*}

