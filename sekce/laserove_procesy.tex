\section{Laserové procesy}

\subsection{Přehled laserových procesů}

Laserové procesy jsou procesy, při kterých jsou zapojeny lasery. Tyto procesy mohou být popsány pomocí různých metod, např. Bornovy aproximace nebo Hartreeho-Fockovy metody.

\subsection{Příklad 3.1.1 - výpočet}
\begin{zadani}
    Yttrium-Aluminium-Granát laser dotovaný neodymem, YAG:Nd3+, 
    má rozdíl energií mezi horní a spodní laserovou hladinou 1,17 eV. 
    Určete vlnovou délku příslušného laserového záření a stanovte
    jaké oblasti spektra odpovídá.
\end{zadani}


Energie vyzářeného fotonu je přímo svázaná s vlnovou délkou takového záření:
\begin{align*}
  \lambda &= \frac{hc}{\Delta E} \\
  \lambda &= \frac{\consthval\constcval}{1,17\cdot \consteval} \\
  \lambda &= \qty{1,06e-6}{\meter}
\end{align*}

Tato vlnová délka odpovídá infračervené oblasti elektromagnetického spektra.


\subsection{Příklad 3.1.4 - výpočet}
\begin{zadani}
    Laserový svazek o průměru d = 0,1 mm  dopadl kolmo na destičku z
    křemíku. Výkon přenášený ve svazku je P = 10 kW.  Stanovte plošnou 
    hustotu výkonu laserového záření po průchodu destičkou o tloušťce 
    x = 2 mm,  použijeme-li záření o vlnové délce \(\lambda\)  = 1,06 µm, součinitel 
    reflexe uvažujme  R = 0,28  a absorpční součinitel \(\alpha(\lambda)\)  = 5 cm-1.
   Rozhodněte, zda na dané vlnové délce lze křemík použít pro optické 
    systémy
\end{zadani}


Zde se kombinují dva jevy, nejprve se část intenzity paprsku odrazí zpět a následně je část pohlcena materiálem. Pro stanovení hustoty výkonu po průchodu systémem musíme zohlednit oba jevy. 

Plošná hustota výkonu na začátku:
\[
    N_{S_{0} } =\frac{P_{0}}{S}=\frac{4P_{0}}{\pi d^2} = \qty{1,27e12}{\watt\per\square\meter} 
\]
Odečtení odraženého výkonu: 
\[
    N_{S_{0} }^\prime = N_{S_{0} } (1-R)=\qty{9,144e11}{\watt\per\square\meter}
\]
Odečtení absorbovaného výkonu: 
\begin{align*}
  N_{S} &= N_{S_{0} }^\prime\cdot \exp(-\alpha(\lambda)x) \\
  N_{S} &= \num{9,144e11}\cdot \exp(-\num{5e2}\cdot \num{2e-3}) \\
  N_{S} &= \qty{3,36e11}{\watt\per\square\meter}
\end{align*}
 
