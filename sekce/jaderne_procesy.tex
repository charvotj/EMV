\section{Jaderné procesy}

\subsection{Příklad 2.2.1 - výpočet}
\begin{zadani}
    Stanovte rychlost s jakou se pohybují  tzv. tepelné neutrony
    při teplotě 25 °C 
\end{zadani}

Vyjdeme zde z kinetické energie neutronu, ta je rovna jeho energii tepelné:
\[
    E=\frac{1}{2}m_{n} v_{n}^{2}=kT 
\]
Z této rovnosti vyjádříme rychlost neutronu v závislosti na termodynamické teplotě:
\begin{align*}
  v_{n} &= \sqrt{\frac{2kT}{m_{n}}} \\ 
  v_{n} &= \sqrt{\frac{2\cdot\constkval(25+273,15)}{\num{1,674e-27}}} \\ 
  v_{n} &= \qty{2216}{\meter\per\second}
\end{align*}


\subsection{Příklad 2.2.7 - výpočet}
\begin{zadani}
    Stanovte kolikrát se zmenší tok tepelných neutronů při průchodu
    destičkou z kadmia a hliníku o tloušťce 1 mm. Účinný průřez pro
    kadmium  \(\sigma_{Cd} \)= 2 500.10-28 m2  a pro hliník  \(\sigma_{Al} \)= 0,21.10-28 m2 
\end{zadani}

Pro vypočtení tohoto příkladu je potřeba stanovit absorbční koeficient obou materiálů:
\begin{align*}
    \alpha &= n\sigma \\\\
    \alpha_{Cd}  &= n_{Cd} \sigma_{Cd} = \num{4,6e28}\cdot \num{2500e-28}= \qty{11500}{\per\meter}\\
    \alpha_{Al}  &= n_{Al} \sigma_{Al} = \num{6,02e28}\cdot \num{0,21e-28}= \qty{1,26}{\per\meter}
\end{align*}

Následně využijeme vztah pro stanovení toku neutronů po průchodu vrstvou materiálu tlustou x m:
\begin{align*}
    \varphi(x) &= \varphi_{0} \cdot \exp(-\alpha x)\\
    \frac{\varphi_{0}}{\varphi(x)} &=\frac{1}{\exp(-\alpha x)}
\end{align*}

Zlomek na posledním řádku přímo odpovídá na naši otázku "kolikrát se zmenší?". Po dosazení hodnot pro oba materiály získáme následující hodnoty:
\begin{itemize}
    \item Po průchodu kadmiem se tok zmenší \num{99009} krát. 
    \item Po průchodu hliníkem pouze \num{1,002} krát.
\end{itemize}





