\section{Elektronové procesy}

\subsection{Přehled elektronových procesů}

Elektronové procesy jsou procesy, při kterých jsou zapojeny elektrony. Tyto procesy mohou být popsány pomocí různých metod, např. Bornovy aproximace nebo Hartreeho-Fockovy metody.

\subsection{Příklad 1.1.2 - výpočet}
\begin{zadani}
    Určete hustotu emisního proudu rozžhaveného wolframového vlákna 
    při teplotě 2 227 °C , působí-li současně elektrické pole o intenzitě 
    E = 2.105 V.m-1
\end{zadani}

V této sekci je uveden výpočet pro příklad 1.1.2. Jsou uvedeny použité metody a vypočtené výsledky.

\subsection{Příklad 1.1.3 - výpočet}
\begin{zadani}
    Určete emisní proud katody vyrobené ve tvaru disku o průměru 10 mm
    z wolframu, vyhřáté na teplotu 2 227 oC. Urychlovací anoda je umístěna 
    ve vzdálenosti  50 mm  od katody a je na ni přivedeno napětí  10 kV
\end{zadani}

V této sekci je uveden výpočet pro příklad 1.1.3. Jsou uvedeny použité metody a vypočtené výsledky.