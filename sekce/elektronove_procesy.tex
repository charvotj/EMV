\section{Elektronové procesy}

\subsection{Příklad 1.1.2 - výpočet}
\begin{zadani}
    Určete hustotu emisního proudu rozžhaveného wolframového vlákna 
    při teplotě 2 227 °C , působí-li současně elektrické pole o intenzitě 
    E = 2.105 V.m-1
\end{zadani}

Nejprve je potřeba vypočítat emisní proud samotného vlákna, k tomu slouží následující vztah:
\[
    j_{eT} = A T^2 \exp\left(-\frac{W}{kT}\right)
\]
Po dosazení dostáváme:
\[
    j_{eT} =\qty{3155}{\ampere\per\square\meter}
\]
Takto vypočtená hodnota odpovídá emisnímu proudu samotného wolframového vlákna, bez působení el. pole. Působením napětí můžeme proudovou hustotu výrazně zesílit, platí zde následující vztah:
\begin{align*}
    j_{eTE} &= A T^2 \exp\left(-\frac{W-\Delta W}{kT}\right) \\
    j_{eTE} &= A T^2 \exp\left(-\frac{W}{kT}\right) \cdot \exp\left(\frac{\Delta W}{kT}\right) \\
    j_{eTE} &= j_{eT} \cdot \exp\left(\frac{\Delta W}{kT}\right)= j_{eT} \cdot \exp\left(\frac{0,44\sqrt{E} }{T}\right)\\
    j_{eTE} &= \num{3155}\exp\left(\frac{0,44\sqrt{\num{2e5}} }{\num{2500}}\right) \\
    j_{eTE} &= \qty{3413}{\ampere\per\square\meter }
\end{align*}



\subsection{Příklad 1.1.3 - výpočet}
\begin{zadani}
    Určete emisní proud katody vyrobené ve tvaru disku o průměru 10 mm
    z wolframu, vyhřáté na teplotu 2 227 oC. Urychlovací anoda je umístěna 
    ve vzdálenosti  50 mm  od katody a je na ni přivedeno napětí  10 kV
\end{zadani}

Vyjdete z proudové hustoty vypočtené v předchozím příkladu, pro stanovení proudu stačí stanovit plochu katody a dosadit do obecně známého vztahu:

\begin{align*}
  I&=j_{eTE} S = j_{eTE} \frac{\pi d^2}{4}\\
  I&=j_{eTE} S = \num{3413} \frac{\pi (\num{10e-3})^2}{4}\\
  I&=\qty{0.268}{\ampere}
\end{align*}

