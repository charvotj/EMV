\section{Ultraakustické procesy}

\subsection{Přehled ultraakustických procesů}

Ultraakustické procesy jsou procesy, při kterých jsou zapojeny zvukové vlny. Tyto procesy mohou být popsány pomocí různých metod, např. Bornovy aproximace nebo Hartreeho-Fockovy metody.

\subsection{Příklad 3.2.4 - výpočet}
\begin{zadani}
    Jak velký výkon bude předán destičce o ploše 1 cm2 ponořené ve 
    vodě, bude-li amplituda kmitů Y = 10-6 m  při kmitočtech 20 kHz a 
    1 MHz, při kolmém dopadu a úplné absorpci vlnění?
\end{zadani}


Výkon akustických vln závisí na prostředí, kde vlna působí (to určuje rychlost šíření vlnění) a také jeho amplituda. Výkon označíme \(P\) a vyjdeme z následujícího vztahu:
\begin{align*}
    P &= \frac{1}{2}\omega^2 Y^2 \rho vS
\end{align*}
Dosazení pro \qty{20}{kHz}:
\begin{align*}
    P &= \frac{1}{2}(2\pi\cdot \num{20e3})^2 (\num{e-6})^2 \num{1000}\cdot \num{1483}\cdot \num{1e-4} \\
    P &= \qty{1,17}{\watt} \\
\end{align*}
Dosazení pro \qty{1}{MHz}:
\begin{align*}
    P &= \frac{1}{2}(2\pi\cdot \num{1e6})^2 (\num{e-6})^2 \num{1000}\cdot \num{1483}\cdot \num{1e-4} \\
    P &= \qty{2,926}{\kilo\watt} \\
\end{align*}

\subsection{Příklad 3.2.7 - výpočet}
\begin{zadani}
    Vypočtěte jak velká část intenzity dopadajícího ultrazvukového vlnění 
    projde do oceli a jak velká část se odrazí při kolmém dopadu 
    akustické vlny na rozhraní  voda – ocel,  resp.  vzduch – ocel
\end{zadani}


Pro rozhraní libovolných dvou prostředí lze stanovit součinitel odrazu, který udává, jak velká část intenzity vlnění se odrazí zpět, zbytek naopak projde. 
Tento součinitel závisí na akustickém vlnovém odporu obou prostředí. 

Výpočet akustického vlnového odporu:
\[
    Z_{a} = \rho v
\]
\begin{align*}
  \text{Vz}&\text{duch:} & \text{Vo}&\text{da:} & \text{Oc}&\text{el:} \\
  Z_{vz} &= \rho_{vz} v_{vz} & Z_{vo} &= \rho_{vo} v_{vo} & Z_{oc} &= \rho_{oc} v_{oc} \\
  Z_{vz} &= \num{1,276}\cdot\num{332} & Z_{vo} &= \num{1000}\cdot\num{1483} & Z_{oc} &= \num{7800}\cdot\num{6000} \\
  Z_{vz} &= \qty{423.6}{\kilo\gram\per\square\meter\per\second} & Z_{vo} &= \qty{1.483e6}{\kilo\gram\per\square\meter\per\second} & Z_{oc} &= \qty{4.68e7}{\kilo\gram\per\square\meter\per\second} \\
\end{align*}

Koeficienty odrazu:
\begin{align*}
  R_{vo-oc} &= \left( \frac{Z_{vo}-Z_{oc}}{Z_{vo}+Z_{oc}} \right)^2 \\
  R_{vo-oc} &= \left( \frac{\num{1,483e6}-\num{4,68e7}}{\num{1,483e6}+\num{4,68e7}} \right)^2 \\
  R_{vo-oc} &= \num{0,881} = \qty{88,1}{\percent}  
\end{align*}
Na rozhraní voda--ocel se odrazí \qty{88,1}{\percent} intenzity vlnění, zbylých \qty{11,9}{\percent} projde dále.

\begin{align*}
    R_{vo-oc} &= \left( \frac{Z_{vz}-Z_{oc}}{Z_{vz}+Z_{oc}} \right)^2 \\
    R_{vo-oc} &= \left( \frac{\num{423,6}-\num{4,68e7}}{\num{423,6}+\num{4,68e7}} \right)^2 \\
    R_{vo-oc} &= \num{1} = \qty{100}{\percent}  
\end{align*}
Na rozhraní vzduch--ocel se odrazí téměř \qty{100}{\percent} intenzity vlnění.
