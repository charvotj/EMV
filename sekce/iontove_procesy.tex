\section{Iontové procesy}

\subsection{Přehled iontových procesů}

Iontové procesy jsou procesy, při kterých jsou zapojeny ionty. Tyto procesy mohou být popsány pomocí různých metod, např. Bornovy aproximace nebo Hartreeho-Fockovy metody.

\subsection{Příklad 1.2.1 - výpočet}
\begin{zadani}
    Srovnejte základní vlastnosti elektronů a iontů.
\end{zadani}


V této sekci je uveden výpočet pro příklad 1.2.1. Jsou uvedeny použité metody a vypočtené výsledky.

\subsection{Příklad 1.2.2 - výpočet}
\begin{zadani}
    Stanovte maximální koncentraci implantovaných iontů bóru do 
    monokrystalu křemíku ve vzdálenosti středního doletu iontů  Rp od 
    povrchu destičky a koncentraci iontů ve vzdálenosti  Rp \(\pm \Delta\)Rp a na 
    povrchu monokrystalu.  \(\Delta\)Rp  je střední kvadratická odchylka. Zjištěný  
    koncentrační profil naznačte graficky. Celková dávka  Q = 1018 m-2 při  
    energie iontů  100 keV
\end{zadani}
t

V této sekci je uveden výpočet pro příklad 1.2.2. Jsou uvedeny použité metody a vypočtené výsledky.